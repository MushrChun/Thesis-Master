\chapter{Introduction}
The "pay-as-you-go" Cloud Computing model has dominated the private data centres market when it comes to web applications and batch processing\cite{bonomi2012fog}. The phenomenon is not out of expectation since Rajkumar Buyya predicted that the Cloud Computing was going to become one of the utilities namely water, electricity, gas and telephony in 2008\cite{buyya2009cloud}. Various modes of Cloud Computing are discussed with the perspective of business as well as appropriate algorithms based on these strategies. Paas, Iaas, SaaS are big names among them.

The world is not static. With the emerging devices connected to the Internet, generating data and asking for intelligent analysis, the burden of Cloud Computing increase sharply. As the data created through the Internet of Things(IoT) is expected to exceed counterparts of human beings, the single layer of Cloud Computing model is challenged \cite{vaquero2014finding}. Latency-sensitive applications get influenced heavier.

Bonomi et al., Cisco researchers, coined a term Fog Computing, tried to through light on this issue. In their concept, a middle layer called fog layer is introduced between the cloud(server-side) and end users(client-side). They argue that this three-layer mechanism may be the antiseptic solution of prosperous IoT. The aim of low latency will be achieved with the bonus of location awareness.

Potential scenarios where the Fog Computing mode shines attract researchers' interests. Given the trend of the rising volume, velocity and various data flow, real-time data analysis and relevant machine learning tasks are paid more attention. As a result, scenarios with the tag of "smart" such as Smart City, Smart Health, Smart Transportation gain popularity \cite{madsen2013reliability}.

Face identification, including face detection and face recognition, is predicted to be in high demand among these smart scenarios. So a Fog Computing based face identification should be proposed to fill in the gap. Considering the limited power of computing and battery, Fog Nodes in the architecture cannot handle the whole traffic alone; Cloud Nodes involve accordingly. A combination model of Fog Nodes and Cloud Nodes is formed, which offers end users a transparent computing layer where the location of the computing is unknown to them.
