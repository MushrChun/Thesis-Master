\chapter{Introduction}
The "pay-as-you-go" Cloud Computing model has dominated the private data centres market when it comes to web application and batch processing.\cite{bonomi2012fog} This is not out of expectation since Rajkumar Buyya predicted that the Cloud Computing was going to become the utilities namely water, electricity, gas and telephony in 2008.\cite{buyya2009cloud} Various modes of Cloud Computing are discussed with the perspective of business as well as relevant algorithms based on these strategies. Paas, Iaas,SaaS are big names among them.

The world is not static. With the emerging devices connected to the Internet, generating data and asking for intelligent analysis, the burden of Cloud Computing increase sharply. As the data created through Internet of Thing is expected to exceed counterparts of human beings, the single layer of Cloud Computing model is challenged. Latency-sensitive applications get influenced heavier.

Bonomi et al., Cisco researchers, coined a term Fog Computing, tried to through light on this issue. In their concept, a middle layer called fog layer is introduced between the cloud(server-side) and end users(client-side). They argue that this three layer mechanism may be the remedial solution of prosperous IoT. The aim of low latency will be achieved with the extra bonus of location awareness.

My research falls on the investigation of the potential of Fog Computing model and finding of undiscovered Fog Computing use cases. Given the trend of raising volume, velocity and various data flow, real-time data analysis and relevant machine learning tasks are focused.