\chapter{Conclusion} \label{chap:conclusion}

This research explains the inception of Fog Computing and discusses the appropriate scenarios based on the background of prosperous IoT devices. Accordingly, face identification attracts attention, and its computing model gets optimised in this research. Fog Computing based face identification architecture is refined after inspecting the previous related work. An aggressive model is proposed where the functionality of Fog Nodes is extended from image pre-processing to full-edged face identification.

A prototype system is implemented to evaluate the performance of the devised device-fog-cloud design. Peak and off-peak modes are composed to mimic different pressure of Fog Nodes. The performance of the Fog Basic mode (an off-peak mode) displays acceptable quality of service, offering confidence to offload face detection from Cloud. The Fog DNN mode (an off-peak mode) presents terrible outcome, leading to a dependency of the Cloud Nodes. Fog Cloud mode (a peak mode) reveals the potential of a combination of Fog Nodes and Cloud Nodes where dynamic allocation mechanism is introduced to overflow tasks to Cloud Nodes in part.

Besides, the controlled experiments consisting of Native mode and Local mode, are conducted to compare the result to Fog Computing based models. In both modes, no Fog Nodes and Cloud Nodes involve. Their awkward performance strengthens the beliefs in Fog based face identification.

The combination of Fog Nodes and Cloud Nodes deliver a transparent computing model to end devices, which benefits to deploy a universal programming model. This observation is regarded as a bonus to this research.