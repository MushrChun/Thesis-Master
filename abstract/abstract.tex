\chapter*{Abstract}

Soaring number of Internet of Things (IoT) devices get connected to the Cloud Computing infrastructure. Generated data from them continuously consumes computational resource for intelligent analysis, leading to huge pressure for Cloud Computing to remain low latency. The face identification, including face detection and face recognition, is one of the typical tasks contributing to the inability of Cloud Computing when low latency and security is considered.

The research proposes an aggressive model where Fog Nodes are made use of to relieve the pressure of Cloud infrastructures. Since Fog Nodes are closer to end devices and more configurable, they extend the power of the Cloud Computing to reach satisfied latency. A prototype is implemented to evaluate the performance of this Fog Computing based face identification system. Five modes are created to distinguish outcome of various scenarios, including peak and off-peak.

A dynamic allocation mechanism is drafted to judge where the computing is run. Cloud Nodes are in demand when the Fog Nodes fail to respond in time. The mechanism help overflow requests from the end nodes to the Cloud Nodes in part.

Finally, a transparent computing model is proposed based on the combination of Fog Nodes and Cloud Nodes.
